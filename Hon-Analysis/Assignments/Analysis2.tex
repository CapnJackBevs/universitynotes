%&pdflatex
\documentclass{article}
\usepackage[utf8]{inputenc}

% Set page geometry.
\usepackage{geometry}
 \geometry{
 a4paper,
 total={170mm,257mm},
 left=20mm,
 top=10mm,
 }

% Packages for maths symbols.
\usepackage{amsmath}
\usepackage{amsfonts}
\usepackage{graphicx}
\usepackage[colorlinks=true, allcolors=blue]{hyperref}

% Various definitions.
%\definecolor{bblue}{rgb}{0.36, 0.54, 0.66}
\def \Z {\mathbb{Z}}
\def \N {\mathbb{N}}
\def \Q {\mathbb{Q}}
\def \R {\mathbb{R}}
\def \C {\mathbb{C}}
\def \E {\mathbb{E}}
\def \P {\mathbb{P}}
\def \> {\Rightarrow}
%\def \bb {\color{bblue}}
\def \thm {\vspace{1ex} \par \begingroup \narrower \noindent}
\def \thmend {\par \endgroup \vspace{1ex}}

% Document Specifics.
\author{William A. Bevington}
\title{Analysis Hand In Two}
\date{}
\begin{document}
\maketitle
%%%%%%%%%%%%%%%%%%%%%%%%%%%%%%
\section*{Question One - Worksheet 3, Q6}
% Show that f(x) = sin(x) is uniformly continuous on R
	\begin{equation}\label{eqn:q1}
		f(x) = \sin(x),\quad x\in\R
	\end{equation}
We wish to show that $\forall\varepsilon>0: \exists\delta>0: \forall x,y\in\R:$ we have that $|x-y|<\delta \Rightarrow |\sin(x)-\sin(y)|<\varepsilon$, note that in the case of uniform continuity  $\varepsilon$ depends only on $\delta$. Since $\sin(x)$ is differentiable on $I=\R$ with a bounded derivative, $\left|\frac{d}{dx}\sin(x)\right| = |\cos(x)| \leq 1$ for any $x\in\R$, I will instead prove that for any differentiable function $g:\R\to\R$ with a bounded derivative is uniformly continuous, the result will then follow.

Since $g:\R\to\R$ has a bounded derivative by assumption we have that the limit
	\begin{equation}\label{eqn:q1}
		g'(y) := \lim_{x\to y} \frac{g(x)-g(y)}{x-y}
	\end{equation}
exists and is bounded by some $M\in\R$ so that $\forall x\in\R: |g'(x)|<M$. In other words; if $|x-y|<\delta$ then $\left|\frac{g(x)-g(y)}{x-y}-g'(y)\right|<\epsilon$. Let $\varepsilon = \frac{M\delta}{1+\delta}$, so we have that $\delta = \frac{\varepsilon}{M-\varepsilon}$ and since $|x-y|<\delta$ 
	\begin{align*}
		\epsilon
		&>\left| \frac{g(x)-g(y)}{x-y}-g'(y)  \right|
		\geq \left| \frac{g(x)-g(y)}{x-y} - M \right|
			&\text{since $\forall x: |g'(x)|<M$}\\
		&\geq \left|\left|\frac{g(x)-g(y)}{x-y}\right|-M\right|
			&\text{by the reverse triangle inequality}\\
		&\Rightarrow M-\epsilon < \left|\frac{g(x)-g(y)}{x-y}\right|<M+\epsilon
			&\text{using $|a-b|<c \Rightarrow -c<a-b<c$}
	\end{align*}
and so we have that $|g(x)-g(y)| < |x-y|(M-\epsilon)< \delta(M-\epsilon) = \frac{\varepsilon}{M-\varepsilon}(M-\varepsilon) = \varepsilon$, that is; $g$ is uniformly continuous. Thus if $g:\R\to\R$ has a bounded derivative on $\R$ then $g$ is uniformly continuous, hence $f(x)=\sin(x)$ is uniformly continuous on $\R$.


%%%%%%%%%%%%%%%%%%%%%%%%%%%%%%
\section*{Question Two - Worksheet 3, Q9}
% Find an example f:(0,1)->R which is cts but not uniformly cts. Where exactly did we use that [a,b] was closed and bounded in the theorem?

We will aim to find two sequences $(s_n)_{n\in\N}$ and $(t_n)_{n\in\N}$ for which $|s_n-t_n|\to0$ but that $\lim_{n\to\infty}|f(s_n)-f(t_n)| \neq 0$ for some continuous function $f$, then by question seven of the workshop $f$ will not be uniformly continuous.

Let $f:(0,1)\to\R:x\mapsto1/x$, then $f$ is continuous on $(0,1)$ since if $|x-y|<\delta$ and $\varepsilon = \frac{\delta}{|xy|}$ then
	\[
		|f(x)-f(y)| = \left|\frac1x - \frac1y\right| 
		= \left|\frac{y-x}{xy}\right| 
		< \frac{\delta}{|xy|} = \varepsilon.
	\]
Now we must show that $f$ isn't uniformly continuous. Let $(s_n)_{n\in\N}$ be a sequence defined by $s_n = 1/2n$ so that $s_n\to 0$ and $\forall n\in\N: s_n\in(0,1/2]\subset(0,1)$, note that $f(s_n) = f(1/2n) = 2n$. Now, let $(t_n)_{n\in\N}$ be a sequence defined by $t_n = 1/3n$ so that $t_n\to0$ as $n\to\infty$, and we have $\forall n\in\N: t_n\in(0,1/3]\subset(0,1)$, and $f(t_n) = 3n$. Then $|f(t_n)-f(s_n)| = |3n-2n|=|n|\to\infty\neq0$, thus $f(x)=1/x$ is not uniformly continuous on $(0,1)$.

The theorem in the workshop stated that if $f:[a,b]\to\R$ is continuous then it's uniformly continuous. This isn't true for open intervals $(a,b)$ since $f$ can asymptotically approach $\pm\infty$ at either of the end points, however if $f$ is defined on $[a,b]$ then in order for $f$ to be defined, $f(a)$ and $f(b)$ must be finite.


%%%%%%%%%%%%%%%%%%%%%%%%%%%%%%
\newpage
\section*{Question Three - Worksheet 4, Q5}
% Prove that
%  (i)   S(x) > 0 for 0 < x <= sqrt(6)
%  (ii)  C(x) > 0 for 0 < x <= sqrt(2)
%  (iii) If 0 <= x <= sqrt(56) and 1-x^2/2! + x^4/4! < 0 then C(x) < 0.
%		Show then that C(8/5)<0
We have that
	\begin{equation}\label{eqn:q3}
		S(x) := \sum_{k=0}^\infty \frac{(-1)^kx^{2k+1}}{(2k+1)!},\quad
		C(x) := \sum_{k=0}^\infty \frac{(-1)^kx^{2k}}{(2k)!}
	\end{equation}
and we wish to show that for $x\in(0,\sqrt{6}]: S(x)>0$ and that for $x\in(0,\sqrt{2}]: C(x)>0$. For $S(x)$, consider equation (\ref{eqn:q3}):
	\begin{align*}
		S(x) &= \sum_{k=0}^\infty \frac{(-1)^kx^{2k+1}}{(2k+1)!} \\
		&= \sum_{k=0}^\infty
			\left(
			\frac{(-1)^{2k}x^{4k+1}}{(4k+1)!} -
			\frac{(-1)^{2k}x^{4k+3}}{(4k+3)!}
			\right) \\
		&= \sum_{k=0}^\infty
			\frac{x^{4k+1}}{(4k+1)!}
			\left(
			1-\frac{x^2}{(4k+2)(4k+3)}
			\right)
	\end{align*}
so if $x\in(0,\sqrt{6}]$ then $x^2\in(0,6]$ and $x^{4k+1}\in(0,36^k\sqrt{6}]$. Notice that $1-\frac{x^2}{(4k+2)(4k+3)}>0$ precisely when $1>\frac{x^2}{(4k+2)(4k+3)}\in \left(0,\frac{6}{(4k+2)(4k+3)}\right]$, and so certainly $1-\frac{x^2}{(4k+2)(4k+3)}>0$. Thus each term $\frac{x^{4k+1}}{(4k+1)!} \left(1-\frac{x^2}{(4k+2)(4k+3)}\right)$ is positive and thus the sum of terms $S(x)>0$ is positive for $x\in(0,\sqrt{6}]$.

Similarly we want to show that $C(x)>0$ for $x\in(0,\sqrt{2}]$. Splitting the sum as we did above we get
	\begin{align*}
		C(x) &= \sum_{k=0}^\infty \frac{(-1)^kx^{2k}}{(2k)!} \\
		&= \sum_{k=0}^\infty 
			\left(
			\frac{(-1)^{2k}   x^{4k  }}{(4k)!} -
			\frac{(-1)^{2k+2} x^{4k+2}}{(4k+2)!}
			\right)\\
		&= \sum_{k=0}^\infty
			\frac{x^{4k}}{(4k)!}
			\left(
			1 - \frac{x^2}{(4k+1)(4k+2)}
			\right),
	\end{align*}
and clearly $\frac{x^{4k}}{(4k)!}$ is positive. Now, $x\in(0,\sqrt{2}] \Rightarrow x^2\in(0,2]$, and since $k\geq0$ we have that $(4k+1)(4k+2)\geq(0+1)(0+2)=2$ and so $1 - \frac{x^2}{(4k+1)(4k+2)}>0$, and since each term is positive the sum of all the terms will be positive too, thus $C(x)>0$ for $x\in(0,\sqrt{2}]$.

Finally we wish to prove that if $x\in[0,\sqrt{56}]$ and $1-x^2/2! + x^4/4! < 0$ then $C(x) < 0$. Rewriting this as a quadratic in $a=x^2$ we have that $a^2 - 12a +24 <0$, and we get critical values $a=6\pm2\sqrt{3}$ and so the critical values for the quartic in $x$ are $\left\{\sqrt{6+2\sqrt{3}},\sqrt{6-2\sqrt{3}},-\sqrt{6+2\sqrt{3}},-\sqrt{6-2\sqrt{3}}\right\}$, which gives (by looking at the graph) that $x^4-12x^2+24<0$ iff $x\in(-\sqrt{6+2\sqrt{3}},-\sqrt{6-2\sqrt{3}})\cup(\sqrt{6-2\sqrt{3}},\sqrt{6+2\sqrt{3}})$, but by assumption we have that $x\in(0,\sqrt{56}]$ so certainly $x>0$ and $\sqrt{6+\sqrt{3}}<\sqrt{56}$ thus $(\sqrt{6-2\sqrt{3}},\sqrt{6+2\sqrt{3}})\subset(0,\sqrt{56})$ and
	\[
		x\in\left(0,\sqrt{56}\right]
		\text{  and  }
		1 - x^2/2! + x^4/4! < 0 
		\quad\Rightarrow\quad
		x\in\left(\sqrt{6-2\sqrt{3}},\sqrt{6+2\sqrt{3}}\right),
	\]
and so $x^2\in\left(6-2\sqrt{3},6+2\sqrt{3}\right)$. Note that 
	\begin{equation}\label{eqn:cos}
		C(x)=
		\sum_{k=0}^\infty 
			\left(
			\frac{x^{4k}}{(4k)!} - \frac{x^{4k+2}}{(4k+2)!}
			\right)
		=
		\sum_{k=0}^\infty   \frac{x^{4k}  }{(4k)!} 
		- \sum_{k=0}^\infty \frac{x^{4k+2}}{(4k+2)!}
	\end{equation}
where $x\in\left(\sqrt{6-2\sqrt{3}},\sqrt{6+2\sqrt{3}}\right)$, and $1-x^2/2!+x^4/4!=0$ so the first two terms of the first sum $1+x^4/4! = x^2/2!$; the first term of the second sum, thus
	\[
		C(x) = 
		\sum_{k=2}^\infty   \frac{x^{4k}  }{(4k)!} 
		- \sum_{k=1}^\infty \frac{x^{4k+2}}{(4k+2)!}
		=
		\sum_{k=2}^\infty \left( \frac{x^{4k}}{(4k)!} 
		- \frac{x^{4k-2}}{(4k-2)!} \right),
	\] 
which is clearly negative, as each term is negative.

Now, we have that for $x\in\left(\sqrt{6-2\sqrt{3}},\sqrt{6+2\sqrt{3}}\right)$ then $C(x)<0$. Since $1.59\approx\sqrt{6-2\sqrt{3}} < 8/5=1.6 < \sqrt{6+2\sqrt{3}}\approx3.08$ by direct calculation, we have that $C(8/5)<0$ by the above.



%%%%%%%%%%%%%%%%%%%%%%%%%%%%%%
\newpage
\section*{Question Four - Worksheet 4, Q6}
% Deduce that there is a unique number w/2 such that sqrt(2)< w/2 < 8/5 such that C(w/2)=0, and that S(w/2)=1.
%Intermediate Value theorem for $cos$. Mean Values thm for $sin = -d/dx(cos(x))$

From the above question we have that $C(8/5)<0$, and $\frac{d}{dx}C(x) = -S(x)$, so $C(x)$ is a continuous funtion on some interval $I$ (in this case $I=\R$, which we proved in the workshop), thus we can use the intermediate value theorem:
	\thm\textbf{Intermediate Value Theorem:}\\
	If $f$ is continuous on some interval $[a,b]$ and takes values $f(a)=y_a$ and $f(b)=y_b$, then for any $y$ between $y_a$ and $y_b$, $\exists c\in[a,b]$ such that $y=f(c)$.
	\thmend
\noindent In our case, $[a,b]$ can be any closed subset of $I=\R$, but we'll choose $[\sqrt{2},8/5]$. We have that $C(8/5)<0$ and that $C(\sqrt{2})>0$ from the above, and thus there must be some $\omega/2\in[\sqrt{2},8/5]$ such that $C(\omega/2) = 0$ since $0$ is between $C(8/5)<0<C(\sqrt{2})$.

Now we wish to show that $S(\omega/2)=1$, note that $\frac{d}{dx}S(x) = C(x)$, and that $\forall x: S(x)\leq1$, and so $1$ is a global maximum for $S(x)$. At any global maximum strictly inside some interval (ie, not one of the endpoints) has a derivative of zero, thus if $S(c)=1$ for some $c$ then $\frac{dS}{dx}|_{x=c} = C(c) = 0$. There is a unique point $\omega/2$ in $(\sqrt{2},8/5)$ (note the lack of end-points) in which $C(\omega/2)=0$ and given that there is a unique point $c\in(\sqrt{2},8/5)$ in which $S(c)=1$ so that $C(c)=0$, it must be that $c=\omega/2$, thus there is a unique $\omega/2\in(\sqrt{2},8/5)$ in which $C(\omega/2)=0$ and $S(\omega/2)=1$.




\end{document}
