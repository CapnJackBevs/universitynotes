%&pdflatex
\documentclass{article}
\usepackage[utf8]{inputenc}

% Set page geometry.
\usepackage{geometry}
 \geometry{
 a4paper,
 total={170mm,257mm},
 left=20mm,
 top=10mm,
 }

% Packages for maths symbols.
\usepackage{amsmath}
\usepackage{amsfonts}
\usepackage{graphicx}
\usepackage[colorlinks=true, allcolors=blue]{hyperref}

% Various definitions.
%\definecolor{bblue}{rgb}{0.36, 0.54, 0.66}
\def \Z {\mathbb{Z}}
\def \N {\mathbb{N}}
\def \Q {\mathbb{Q}}
\def \R {\mathbb{R}}
\def \C {\mathbb{C}}
\def \E {\mathbb{E}}
\def \P {\mathbb{P}}
\def \> {\Rightarrow}
%\def \bb {\color{bblue}}
\def \thm {\par \begingroup \narrower \noindent}
\def \thmend {\par \endgroup}

% Document Specifics.
\author{William A. Bevington}
\title{Analysis Hand In One}
\date{}
\begin{document}
\maketitle

%%%%%%%%%%%%%%%%%%%%%%%%%%%%%%
\section*{Question One}
	\begin{equation}\label{eqn:q1def}
	f_n(x) = \frac{x^n}{1+x^n}
	\end{equation}
To show that (\ref{eqn:q1def}) converges pointwise we must establish that there is a limit-function $f$ where $\forall\epsilon>0:$ $\exists N\in\N$ such that $\forall n\in N : |f_n(x)-f(x)|<\epsilon$. We break this down into the four cases where $x=0$, $x=1$, $x\in(0,1)$ and $x>1$. If $x=0$ then $f_n(x) = 0$ for all $n$, and so $f_n$ clearly converges to $0$. If $x=1$ then $f_n(x) = \frac12$ for all $n$, so clearly $f_n\to\frac12$.

If $x\in(0,1)$ then $x^n\to0$ as $n\to\infty$ so I claim that $f_n(x)\to0$ as $n\to\infty$ also. Notice that $\forall x\in(0,1): |\frac{x^n}{1+x^n}|\leq |\frac{x^{n}}{1}| = |x^n|$, so letting $\epsilon = x^n$ we see that $\forall x>0:$ $\exists N\in\N$ such that $\forall n>N: |f_n(x)-0|\leq |x^n|=\epsilon$.

If $x\in(1,\infty)$ then $|\frac{x^n}{1+x^n} - 1| = |\frac{x^n}{1+x^{n}} - \frac{1+x^n}{1+x^n}| = |\frac{-1}{1+x^n}| \leq \frac{1}{x^n}$. Thus if we let $\epsilon = \frac1{x^n}$ then, $\forall x\in(1,\infty):\forall\epsilon>0: \exists N\in\N$ such that $\forall n>N: |f_n(x) - 1| < \epsilon$. In other words, $f_n\to 1$ pointwise on $(1,\infty)$, and so
	\[
		f_n\to f =
		\begin{cases}
			0       & x\in[0,1) \\
			\frac12 & x = 1 \\
			1       & x\in(1,\infty)
		\end{cases}
	\]

Next, I claim that $f_n\to f$ uniformly on $x\in[0,a)$ if $0\leq a<1$. First I'll show that for intervals $[0,b)$, where $b\geq1$, $f_n$ does not converge uniformly. We must calculate $\lim_{n\to\infty}\sup_{x\in[0,b)}|f_n(x)|$, and show that this does not equal zero. We need only concern ourselves with $b>1$ since if $b=1$ then we can choose $x=b=1$ to get $f_n(1)=\frac12$ and $\forall x\in[0,1): f_n(x)\to0$, so $f_n$ cannot converge uniformly for $b=1$.

We have that $\forall x\in[0,b): x<b$ and if $n>1$ then, as $b>1$ we know that $x^n < b^n$. I will show than $b^n$ is the supremum by contradiction: if $1<M<b^n$ is another upper bound to $x^n$ for $M\in\R$ then $M^\frac1n < b$, so $M\in[0,b)$ and so isn't an upper bound; contradiction! Thus $\sup_{x\in[0,b)}|f_n| = b^n$. Now, $b>1$, so $b^n>1$ and so clearly $\lim_{n\to\infty}\sup_{x\in[0,b)}|f_n| = \lim_{n\to\infty}b^n \neq 0$, thus for $b>1$ we have that $f_n$ cannot converge uniformly.

Finally, I will prove that for $a<1$, $f_n\to f$ uniformly for $x\in[0,a)$. Note that $\forall x\in[0,a): x<a$ so $x^n < a^n$ and $\lim_{n\to\infty}a^n = 0$ (since $a<1$). Letting $\epsilon=a^n$ we have that
	\[
		\forall \epsilon>0: \exists N\in\N: \forall x\in[0,a):
		|f_n(x)-0| = \left|\frac{x^n}{1+x^n}\right| \leq \left|\frac{a^n}{1}\right| = \epsilon,
	\]
and so $f_n\to 0$ uniformly on $[0,a)$ so long as $0\leq a<1$.


%%%%%%%%%%%%%%%%%%%%%%%%%%%%%%
\section*{Question Two}
	\begin{equation}\label{eqn:q2}
		f_n(x) = nx(1-x^2)^n, \quad x\in[0,1].
	\end{equation}
We will begin by using the M-test to show that $g(x)=\sum_n f_n(x) < \infty$, then deduce that $f_n\to f$ pointwise for some $f$.

To begin, we will use the Ratio test on $f_n$ to test convergence of $\sum_n f_n$ on the interval $(0,1)$, noting that $f_n$ converges pointwise for the endpoints $x=0,1$ since $\forall x: f_n(0)=f_n(1)=0$, and certainly $|f_n(0)-0|=0<\epsilon$ for any $\epsilon>0$, thus $f_n\to0$ for $x=0,1$. Testing
	\[
		\frac{f_{n+1}}{f_n} = \frac{(n+1)x(1-x^2)^{n+1}}{nx(1-x^2)^n} = \frac{n+1}{n}(1-x^2),
	\]
we see that $L:=\lim_{n\to\infty}\left|\frac{f_{n+1}}{f_n}\right| = 1-x^2$, hence $0<L<1$ as $x>0$, so $\sum_n f_n$ converges by the ratio test. By the Divergence Theorem, then, we have that $\lim_{n\to\infty}f_n = 0$, and hence $f_n\to0$ at least pointwise.

Consider $\int_0^1 f_n(x)dx = \int_0^1 nx(1-x^2)^n = \left[-\frac12\frac{n}{n+1}(1-x^2)^{n+1}\right]_0^1 = \frac12\frac{n}{n+1}$. Since $f_n\to0$ as $n\to\infty$ we have that
	\[
		\int_0^1 \lim_{n\to\infty}(f_n(x))dx = \int_0^1 f(x)dx = \int_0^1 0 dx = 0 
		\neq 
		\frac12 = \lim_{n\to\infty}\left(\frac12\frac{n}{n+1}\right) 
		= \lim_{n\to\infty}\int_0^1 f_n(x)dx,
	\]
but if $f_n\to f=\lim_{n\to\infty}f_n$ uniformly then $\forall x: \lim\int f_n dx = \int\lim f_n dx$, and so clearly the convergence cannot be uniform on $[0,1]$.

Now we test to see whether (\ref{eqn:q2}) converges uniformly on $[a,1]$ for $a\in(0,1)$. I claim that it does not uniformly converge,  which I'll show by proving that $\lim_{n\to\infty}\sup_{x\in[a,1]} |f_n(x)|\neq0$. Differentiating (\ref{eqn:q2}) we get that $\frac{d}{dx}f_n(x) = n(1-x^2)^n +n^2x(-2x)(1-x^2)^{n-1}$, so 
	\begin{equation}\label{eqn:unif}
		\frac{df_n}{dx} = 0\Rightarrow 1-x^2 = 2nx^2,
	\end{equation}
where I've divided through by $n(1-x^2)^{n-1}$, which means I must assume $x\neq 1$, however if $x=1$ then $f_n(1)=0$ so we can see if this is a maximum later. so by (\ref{eqn:unif}) we have that $(2n+1)x^2=1$, and so $x=\sqrt{\frac1{2n+1}}$ produces the maximum value of $f_n(x)$, where I took the positive root since $x\in[a,1]\subset[0,1]$. This gives
	\begin{align*}
		f_n\left(\sqrt{\frac{1}{2n+1}}\right)
		&= n\sqrt{\frac{1}{2n+1}}\left(1-\left(\sqrt{\frac{1}{2n+1}}\right)^2\right)^n \\
		&= \frac{n}{\sqrt{2n+1}}\left(\frac{2n}{2n+1}\right)^n \\
		&= \frac{2^nn^{n+1}}{(2n+1)^{n+\frac12}} \\
		&\geq \frac{2^nn^{n+1}}{(2n+0)^n} \\
		&= n.
	\end{align*}
Thus $n\leq\lim_{n\to\infty}\sup_{x\in\R}(f_n(x))=\infty$ and so is certainly not zero, and hence $f_n$ doesn't converge uniformly.


%%%%%%%%%%%%%%%%%%%%%%%%%%%%%%
\section*{Question Three}
Knowing that $f_n\to f$ uniformly, and that $x_n\to x$ as $n\to\infty$ we aim to show that $f_n(x_n)\to f(x)$, that is; that $\lim_{n\to\infty}(f_n(x_n)) = f(x)$.

Since $x_n\to x$ we know that $\forall\varepsilon_x>0: \exists N_x\in\N$ such that $\forall n>N_x: |x_n-x|<\varepsilon_x$, we also know that $f_n\to f$ uniformly means that
	\begin{equation}\label{eqn:uniform}
		\forall\varepsilon_f>0: \exists N_f\in\N: \forall n>N_f: \forall x\in\R:
		|f_n(x)-f(x)|<\varepsilon_f.
	\end{equation}
And, finally we are told that each $f_i$ is continuous, so that
	\begin{equation}\label{eqn:cts}
		\forall i: \forall\varepsilon_i>0: \exists\delta>0: \forall x,y\in\R:
		|x-y|<\delta \Rightarrow |f_i(x) - f_i(y)|<\varepsilon_i,
	\end{equation}
and now we just use the triangle inequality, let $N=\max\{N_i,N_f,N_x\}$ and $\varepsilon =2\min\{\varepsilon_i,\varepsilon_f,\varepsilon_x\}$. If we let $\delta=\varepsilon_x$ then $\exists N\in\N: |x-x_n|<\delta$ then:
	\begin{align*}
		|f_n(x_n) - f(x)| &= |f_n(x_n) -f_n(x) + f_n(x)-f(x)| &\\
		&\leq |f_n(x_n) - f_n(x)| + |f_n(x)-f(x)| &\textrm{by the Triangle Inequality}\\
		&< \varepsilon_i + \varepsilon_f  < \varepsilon 
		&\textrm{by (\ref{eqn:uniform}) and (\ref{eqn:cts})}.
	\end{align*}
Where we've used that $|f_n(x_n)-f_n(x)|<\varepsilon_i$ since $f_n$ is continuous, and that $|f_n(x)-f(x)|<\varepsilon_f$ by the uniform convergence of $f_n\to f$. Thus we get that:
	\[
		\forall\varepsilon>0 :\exists N\in\N: \forall n>N:\ 
		|f_n(x_n) - f(x)|<\varepsilon,
	\]
and thus $f_n(x_n) \to f(x)$ as $n\to\infty$.



\end{document}
