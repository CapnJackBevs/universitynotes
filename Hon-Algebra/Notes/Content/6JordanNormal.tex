\noindent\dotfill
\begin{multicols}{2}
\section*{Chapter Six - Jordan normal form}
\begin{enumerate}
    \item{\vspace{-1ex}Calculate the eigenvalues $\lambda_1,\dots,\lambda_s$ along with geometric $\mu_1,\dots,\mu_s$ and algebraic $m_1,\dots,m_s$ multiplicities,}
    %
    \item{
    Compute corresponding eigenspaces
    \[E^k_\lambda=\{\v\in V:(A-\mathbb{I}\lambda)^k\v=0\}.\]}
    %
    \item{Compute the following, and draw the chart on the right:
      \begin{align*}
          d_1 &= \dim(E_\lambda^1) 
            &&\underbrace{\Box \Box \Box \Box \Box}_{d_1\text{ boxes}}\\
          d_2 &= \dim(E_\lambda^2) - \dim(E_\lambda^1)
            &&\underbrace{\Box \Box \Box \Box}_{d_2\text{ boxes}}\\
          &\vdots && \quad\quad\vdots \\
          d_k &= \dim(E_\lambda^k) - \dim(E_\lambda^{k-1})
            &&\underbrace{\Box \Box \Box}_{d_k\text{ boxes}}\\
      \end{align*}}
    %
    \item{Start at the bottom of the diagram, filling row $k$ with the linearly independent eigenvectors in $E_\lambda^k$ which \textbf{are not in} $E_\lambda^{k-1}$. Each time you fill in a box with a vector $\v_k$, fill in every box above with the vectors $\v_{k+1}=(A-\mathbb{I}\lambda)^k\v$ until you reach the top.}
    %
    \item{Repeat steps two to four with different eigenvalues until the diagram is full. Then $Q$ is the matrix whose columns are the top-left vector followed by the vectors below it so the Jordan-normal form is $J=Q^{-1}AQ$.}
    %
    \item{In fact you \textbf{needn't calculate $Q$}. Each column of the diagram is a Jordan block - easy!}
\end{enumerate}

\end{multicols}