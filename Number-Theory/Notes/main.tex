\documentclass{article}
\usepackage{style}
\usepackage[utf8]{inputenc}
\title{Number Theory}
\begin{document}
\maketitle

\section*{Basics}
\begin{theorem}[Division Algorithm]
Let $a,b\in\Z$ and $b>0$, then there exist unique $q,r\in\Z$ with $r\in[0,b)$ such that
    \[
    a = qb + r.
    \]
\end{theorem}

\begin{theorem}[Bezout]
If $d=\gcd(a,b)$ then there exist $m,n\in\Z$ such that
    \[
    ma + nb = d.
    \]
\end{theorem}

\begin{definition}
Let $a,b,u\in\Z$ then
    \begin{itemize}
        \item If $u|1$ then $u$ is a \textbf{unit},
        \item if $a=cd\implies c=1$ or $d=1$ with $a$ a non-unit then $a$ is \textbf{irreducible}
        \item If $a|cd\implies a|c$ or $a|d$ then $a$ is \textbf{prime}.
    \end{itemize}
\end{definition}


\section*{Gaussian Integers}
\begin{definition}[Gaussian Integers]
    \[
    \Z[i] := \{a+bi\,:\,a,b\in\Z\}.
    \]
\end{definition}

\begin{definition}
A Gaussian integer $z\in\Z[i]$ is a \textbf{unit} if $\exists z'\in\Z[i]: z\cdot z'=1$.
\end{definition}

\begin{definition}
Let $a,b,u\in\Z[i]$ then
    \begin{itemize}
        \item If $u|1$ then $u\in U$ is a \textbf{unit} $U=\{\pm1,\pm i\}$,
        \item if $a=cd\implies c\in U$ or $d\in U$ with $a$ a non-unit then $a$ is \textbf{irreducible}
        \item If $a|cd\implies a|c$ or $a|d$ then $a$ is \textbf{prime}.
    \end{itemize}
\end{definition}

\begin{theorem}
A Gaussian integer $z\in\Z[i]$ is prime iff it's irreducible.
\end{theorem}


\section*{...}
\begin{theorem}
If $p= 1\mod 4$ for prime $p$ then $\exists a,b\in\Z: p=a^2+b^2$.
\end{theorem}

\begin{theorem}[Wilson]
If $p$ is prime then it divides $(p-1)!+1$.
\end{theorem}

\section*{Linear Systems}
Really cannot be bothered to do this right now.
\subsection*{Linear Diophantine equations}
\subsection*{Chinese Remainder theorem}

\section*{Primitive roots}
\begin{theorem}[Fermat's Little Theorem]
Let $p$ be prime, then for any $a\in\{1,2,\dots,p-1\}$ we have
    \[
    a^{p-1} = 1\mod p.
    \]
\end{theorem}

\begin{definition}
If $\xi^n\neq1 (\mod p)$ for all $n\in\{1,\dots,p-1\}$ and $\xi^{p-1}=1$ then we say that $\xi$ is a \textbf{primative root}.
\end{definition}

\section*{Legendre Symbol}
\begin{definition}
Let $p>0$ be prime and $a\in\Z$ not be divisable by $p$, then
    \[
    \left(\frac{a}{p}\right)=
    \begin{cases}
    1 &\text{if } \exists r\in\Z: a=r^2\mod p \\
    -1&\text{otherwise}
    \end{cases}
    \]
\end{definition}

\begin{theorem}
Let $p>0$ be prime and $a\in\Z$ not be divisable by $p$, then
\begin{itemize}
    \item $\left(\frac{a}{p}\right) = a^{\frac{p-1}{2}} \mod p$
    \item $\left(\frac{ab}{p}\right)=\left(\frac{a}{p}\right)\left(\frac{b}{p}\right)$
    \item $\left(\frac{ab}{p}\right)=\left(\frac{a-p}{p}\right)$
\end{itemize}
\end{theorem}

\begin{theorem}[Quadratic Reciprocity]
Let $p,q$ be distinct odd primes:
    \begin{align*}
        \text{if }p=1\,\mathrm{mod}\ 4\text{ or }q=1\,\mathrm{mod}\ 4\text{ then}\quad
          &\left(\frac{p}{q}\right)=\left(\frac{q}{p}\right)\\
        \text{if }p=q=3\,\mathrm{mod}\ 4\text{ then}\quad
          &\left(\frac{p}{q}\right)=-\left(\frac{q}{p}\right)
    \end{align*}
\textsc{TIP:} To remember this use that $pq=1\mod 4$ (if one of $p,q=3\mathrm{mod}\ 4$ then you can't use this theorem).
\end{theorem}

\section*{Euler's totient function}

\begin{definition}
\textbf{Euler's totient function} is defined by
    \[
    \varphi(n) := |\{z\in\Z\,:\,z<n\text{ and }\gcd(z,n)=1\}|.
    \]
\end{definition}

\begin{theorem}[Euler]
Let $a,n\in\Z$ then
    \[
    a^{\varphi(n)} = 1 \mod n.
    \]
\end{theorem}

\begin{theorem}
If $m,n\in\Z$ are coprime then $\varphi(mn)=\varphi(m)\varphi(n)$.
\end{theorem}

\begin{theorem}
A number $n\in\Z$ can only be a primative root if it is of the form
    \[
    2,4,p^k,\text{ or }2p^k
    \]
for $p$ an odd prime.
\end{theorem}

\begin{theorem}[Gauss Lemma]
Let $p$ be an odd prime and $a\in\Z$ not be divisable by $p$. Let $S'$ be the set of remainders of the values $\{\frac{a}{p},\frac{2a}{p},\dots,\frac{p-1}{2p}a\}$ whose values are in the range $(0,p)$ then
    \[
    \left(\frac{a}{p}\right) = (-1)^{|S'|}.
    \]
\end{theorem}

\begin{theorem}
Let $n=p_1^{k_1}p_2^{k_2}\dots p_m^{k_m}$, then
    \[
    \varphi(n) = (p_1^{k_1} - p_1^{k_1-1})(p_2^{k_2} - p_2^{k_2-1})\dots(p_m^{m_1} - p_m^{k_m-1}).
    \]
\end{theorem}

%%%%%%%%%%%%%%%%%%%%%%%%%%%%%%%%%%%%%%%%%%%%%%%%%%%%%%%%%%%%%%%%%%%%%%%%%%%%%%%%
\newpage
\section*{From previous notes}
\begin{theorem}
If $x_0$ is a solution to $ax=c\mod y$ iff $\gcd(x_0,y)|c$, then every solution is of the form
    \[
    x = x_0 -\frac{y}{\gcd(x,y)}t,\quad t\in\N.
    \]
\end{theorem}

\begin{theorem}[Hensel]
Suppose $f(x)\in\Z[x]$ and $r,k\in\Z$ with $f(r)=0\mod p^k$ for $p$ prime. Then solutions to $f(x)=0\mod p^k$ are found by:
    \begin{itemize}
        \item{If $f'(r)=0\mod p$ then;}
        \begin{enumerate}
            \item{if $f(r)=0\mod p^k$ then the $p$ solutions are given by $s = r + tp^{k-1}$ with $t\in[0,p-1]$,}
            \item{if $f(r)\neq0\mod p^k$ then there are \textsc{no solutions}.}
        \end{enumerate}
        \item{if $f'(r)\neq0\mod p$ then there is a \textsc{unique} solution given by $s = r-f(r)\big(f'(r)\big)^{-1} \mod p^k$.}
    \end{itemize}
\end{theorem}
\textsc{REMARK:} The above can be considered, in the first case, as extending $f(x)$ using a kind of Taylor series $f(a+b) = f(a)+f'(a)b+\frac{f''(a)}{2!}b^2+\dots$ with $b=p^{k-1}$. In the second case this is somewhat an analogy of Newton's method of successive approximations $x_{n+1}=x_n + \frac{f'(x_n)}{f(x_n)}$.

\begin{theorem}[Quadratic Recipricocy II]
Let $p,q$ be prime, then
    \[
    \left(\frac{p}{q}\right)\left(\frac{q}{p}\right) 
    = (-1)^{\frac{p-1}{2}}(-1)^{\frac{q-1}{2}}.
    \]
\end{theorem}

\end{document}